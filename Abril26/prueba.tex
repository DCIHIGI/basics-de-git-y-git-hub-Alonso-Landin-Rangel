\documentclass{article}
\usepackage[utf8]{inputenc}

\title{Archivo prueba}
\author{Alonso Landín}
\date{Abril 2021}

\begin{document}

\maketitle

\section*{Sección 1}


Se define la \textbf{derivada} de una función $f(x)$ en el punto $x_0$ como:

$$ \lim_{h \to 0}\frac{f(x_0+h)-f(x_0)}{h} $$

La función derivada de $f(x)$ se representa mediante distintas notaciones, como $\frac{df}{dx}$, $f'(x)$, etc.

\section*{Sección 2}


Otra cantidad muy importante asociada a una función que se puede definir mediante un límite es la \textbf{integral} entre $a$ y $b$, definida como:

\begin{equation} \label{eq:1}
\lim_{\Delta x_{i,max} \to 0}f(x_i)\Delta x_i
\end{equation}
siendo $\Delta x_i$ la distancia entre dos puntos de una partición del intervalo $[a,b]$.


El límite en la ecuación \ref{eq:1} se representa por el símbolo

$$ \int_a^b f(x)dx $$
\end{document}
